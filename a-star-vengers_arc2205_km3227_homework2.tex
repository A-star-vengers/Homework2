\documentclass[10pt]{article}

\begin{filecontents*}{\jobname.bib}
  @article{Bessey2010,
   author = {Bessey, Al and Block, Ken and Chelf, Ben and Chou, Andy and Fulton, Bryan and Hallem, Seth and Henri-Gros, Charles and Kamsky, Asya and McPeak, Scott and Engler, Dawson},
   title = {A Few Billion Lines of Code Later: Using Static Analysis to Find Bugs in the Real World},
   journal = {Commun. ACM},
   volume = {53},
   number = {2},
   year = {2010},
   pages = {66--75},
   url = "http://dx.doi.org/10.1145/1646353.1646374",
   publisher = {ACM},
   address = {New York, NY, USA},
  }
  
  @inproceedings{Rutar2004,
   author = {Rutar, Nick and Almazan, Christian B. and Foster, Jeffrey S.},
   title = {A Comparison of Bug Finding Tools for Java},
   booktitle = {Proceedings of the 15th International Symposium on Software Reliability Engineering},
   series = {ISSRE '04},
   year = {2004},
   pages = {245--256},
   numpages = {12},
   url = {http://dx.doi.org/10.1109/ISSRE.2004.1},
   publisher = {IEEE Computer Society},
   address = {Washington, DC, USA},
  }


\end{filecontents*}

\usepackage[left=1in,right=1in,bottom=1in,top=1in]{geometry}
% \usepackage{fancyvrb}
\usepackage{listings}
\lstset{
basicstyle=\small\ttfamily,
columns=flexible,
breaklines=true
}


\usepackage{mdframed}
\usepackage{commath}
\usepackage{hyperref}
\usepackage{amssymb}
\usepackage{amsthm}
\usepackage{setspace}
\usepackage{graphicx}
\usepackage{tikz}

\usetikzlibrary{external}
\usetikzlibrary{plotmarks}
\usepackage{pgfplots}
\usepackage{standalone} % http://tex.stackexchange.com/questions/11583/insert-a-tikz-picture-from-an-external-file

\usepackage{multirow}
\usepackage{subfloat}
\usepackage{caption}
% \usepackage[nolists]{endfloat}

\interfootnotelinepenalty=10000
\usepackage[hyperref, doi, 
            style      =authoryear-icomp,
            bibstyle   =authoryear,
            url        =true, % false, 
            natbib     =true,
            maxbibnames=99]{biblatex}

\bibliography{\jobname.bib}
\renewbibmacro{in:}{}
\renewbibmacro{in:}{}

\hypersetup{
    bookmarks=true,         % show bookmarks bar?
    unicode=false,          % non-Latin characters in Acrobat’s bookmarks
    pdffitwindow=false,     % window fit to page when opened
    pdfstartview={},    % fits the width of the page to the window
    pdftitle={},    % title
    pdfauthor={Kyle Matoba, km3227@columbia.edu},     % author
    pdfsubject={},   % subject of the document
    pdfcreator={},   % creator of the document
    pdfproducer={}, % producer of the document
    pdfkeywords={} {} {}, % list of keywords
    pdfnewwindow=true,      % links in new window
    colorlinks=true,       % false: boxed links; true: olored links
    linkcolor=black,          % color of internal links
    citecolor=black,        % color of links to bibliography
    filecolor=black,      % color of file links
    urlcolor=black         % color of external links
}

\theoremstyle{remark}

\newtheorem{definition}{Definition}
\newtheorem{axiom}{Axiom}
\newtheorem{remark}{Remark}
\newtheorem{assumption}{Assumption}
\newtheorem{example}{Example}
\newtheorem{theorem}{Theorem}
\newtheorem{lemma}{Lemma}
\newtheorem{corollary}{Corollary}
\newtheorem{result}{Result}
\newtheorem{fact}{Fact}
\newtheorem{hypothesis}{Hypothesis}

\numberwithin{equation}{section}

\newcommand{\tikzcircle}[2][red,fill=red]{\tikz[baseline=-0.5ex]\draw[#1,radius=#2] (0,0) circle ;}%
\newcommand{\blackcircle}{\tikzcircle[black, fill=black]{1.5pt}}
\newcommand{\whitecircle}{\tikzcircle[black, fill=white]{1.5pt}}


\newcommand{\remarkname}{Remark}
\newcommand{\lemmaname}{Lemma}
\newcommand{\assumptionname}{Assumption}
\newcommand{\definitionname}{Definition}
\newcommand{\hypothesisname}{Hypothesis}

\def\subsubsectionautorefname{Section}  
\def\subsectionautorefname{Section}  
\def\sectionautorefname{Section}  

\def\equationautorefname{Equation \hspace{-.15cm}}  

% http://www.tug.org/applications/hyperref/manual.html#TBL-23
\DeclareMathOperator*{\argmin}{arg\,min}
\DeclareMathOperator*{\argmax}{arg\,max}
\DeclareMathOperator*{\minimize}{minimize}
\DeclareMathOperator*{\maximize}{maximize}

\newcommand{\sign}{\textnormal{sign}}
\newcommand{\sd}{\textnormal{sd}}
\newcommand{\diag}{\textnormal{diag}}
\newcommand{\tr}{\textnormal{tr}}
\newcommand{\nnaive}{na\"{\i}ve\ }

\newcommand{\floor}[1]{ \left\lfloor #1 \right\rfloor}
\newcommand{\ceiling}[1]{ \left\lceil #1 \right\rceil}
\renewcommand{\hat}{\widehat}
\renewcommand{\tilde}{\widetilde}
\renewcommand{\d}{\textsc{d}}
\newcommand{\cov}{\textsc{cov}}
\newcommand{\var}{\textsc{var}}
\setlength{\parindent}{0cm}

\begin{document}

\title{Practice with Development Technologies: Homework 2 }

\author{
Andrew Calvano\thanks{Email: \href{mailto:arc2205@columbia.edu}{\texttt{arc2205@columbia.edu}}} \\ 
Kyle Matoba\thanks{Email: \href{mailto:km3227@columbia.edu}{\texttt{km3227@columbia.edu}}}
}
\date{}
\maketitle

% You should work together with your pair partner from your team; each pair in the team needs to do 
% this assignment and submit separately.
% Build a toy system using your team's selected programming language, development framework, and 
% other technologies. You can start with some existing program, e.g., from your static analysis 
% assignment, that you port into the framework or you can write something new.  It does not have 
% to do much. It just needs to use your development framework (which means it needs a front-end,
% a couple screens are fine), build, and "pass" static analysis and a few simple tests.  It should 
% store something persistently and be able to read and update that data again later. Post your code, 
% build scripts, test cases, etc. on github. Submit a single file, 1-2 pages, that tells us the url 
% for this github repository, briefly describes your toy system, and discusses your experience using
% each part of your team's chosen technology.  Did you run into any challenges?  Was there anything 
% you wanted to do but couldn't get working?
% The name of this file should include both your team name and pair unis. 
% You should submit as soon as possible after you are done, you do not need to wait for the deadline.

\section{} 

\subsection{Executive Summary}

We got the following technologies to all work together nicely to meet the specs of the assignment:

\begin{itemize} 
\item flakes8 (pep8 [style checker] and pyflakes [static analyser] and a cyclomatic complexity checker)
\item flask (Jinja2 [templating engine] and Werkzeug [Web Server Gateway Interface utility])
\item SQLAlchemy [SQL toolkit and object-relational mapper] and sqlite [SQL database]
\item unittest [unit testing] and Flask-testing [unit testing tools that ease automatically exercising the logic]
\item virtualenv, pip, and Travis CI
\end{itemize} 

The code is at: \url{https://github.com/A-star-vengers/Homework2}. As may be evident from the commit history, the story of the code can roughly be stated as: ``Andrew put up a bare-bones flask app,\footnote{i.e. what might be downloaded from \url{https://github.com/realpython/discover-flask/tree/part11}} we added a simple state that is persisted across sessions, by user (just a string, with no checking of any kind), added a Travis CI \texttt{.travis.yml} file, and some flakes8 checks, which turned up some small-ish problems (some pep8, some pyflakes) that we were obliged to fix in order to get a clean build. 

\subsection{Bare bones run} 

If you want to run our app, you can look at the build steps at \url{https://travis-ci.org/A-star-vengers/Homework2}, or assuming you have python 2.7 + virtualenv, a C compiler (needed for some of the authentication packages), and git installed, you can do something like:

\begin{verbatim} 
git clone https://github.com/A-star-vengers/Homework2.git
cd Homework2
virtualenv venv
source venv/bin/activate
pip install -r requirements.txt
python run.py
\end{verbatim} 

Then navigate to localhost, port 5000 (or whatever URL is displayed in the console in your browser). To run the tests, run \texttt{python test.py}.

\subsection{More Detail}
As we detailed in our first submission, we propose building a web app to construct and play crossword puzzles socially. Our technology is built atop Flask, using SQLAlchemy for database management. This project consisted of a simple Flask server, along with necessary front-end pages, that allow a user to register an account, sign in, sign out, and change a state that persists across sessions (meant to mock up general business logic to follow later). The code is kept under version control on GitHub, and is built on every commit using Travis CI, this means that (starting from a clean state), the code is cloned, necessary dependencies are downloaded and compiled, the code undergoes static analysis and a style check, and unit tests are run. 

This assignment was well thought out: it obliged us to separate the framework from the logic of our particular app and thereby gave us a great start on our eventual program, this is really the essence of the difference between an advanced course in software development and a first course in coding. Nothing was especially challenging, and we were able to pretty straightforwardly get everything we wanted working (Andrew had some previous experience with Flask). We learned a bit about technologies that we had not used before, for example we initially used Pyflakes and pep8 as protoypical static checking. However, we found that pyflakes raised false positives, and moreover did not have a simple primitive way to overrule (build-blocking) flags.\footnote{Leading some, to suggest entirely disgusting workarounds. For example, \url{http://chase-seibert.github.io/blog/2013/01/11/bypass_pyflakes.html}.} In navigating this hiccup, we learned about a tool that no one in our team had initially suggested, flake8, which combines both (and another) checkers, and more to our point, allows lint-style structured comments instructing the checker to forgive the ordinary checks (the ``\verb|# flake8: noqa|'' in \url{https://github.com/A-star-vengers/Homework2/blob/master/app/__init__.py#L5}). We also learned how to use Travis CI, and moreover learned that it is amazingly lightweight and simple, compared to Jenkins.


\subsection{Next steps} 
Our app is meant to maximally spec out different technologies we will have to use in the final project, not to exhibit any nontrivial logic. So, we succeeded in covering a really quite a nontrivial workflow: users can create accounts and log in securely, and update a state persistent across sessions, the code is built and tested automatically using powerful CI tools. However, it remains to fill in this skeleton with the logic necessary to fully implement our project. 

% \nocite{Rutar2004}
% \pagebreak 
\begin{spacing}{0.9}
\printbibliography
\end{spacing}

\end{document}

